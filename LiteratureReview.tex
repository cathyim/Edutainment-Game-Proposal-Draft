\documentclass[10pt,twocolumn]{article} 

\usepackage{oxycomps} % use the main oxycomps style file

\bibliography{references}

\pdfinfo{
    /Title (Literature Review Part 1)
    /Author (Catherine Yim)
}

\title{"Literature Review"}

\author{Catherine Yim}
\affiliation{Occidental College}
\email{cyim@oxy.edu}

\begin{document}

\maketitle

\section{Problem Context}
According to the article \citetitle{cleggtrayhurn2000} \cite{cleggtrayhurn2000}
, although women’s overall participation in higher education exceeded that of men’s in the UK, women are underrepresented in IT-related occupations. More recently, Girls Who Code's - a non-profit organization dedicated to promoting gender diversity and aiding young women who are historically underrepresented in computer science fields - \citetitle{girlswhocode2021} \cite{girlswhocode2021} page claimed that the gap continues to widen in computer science specifically. “In 1995, 37[percent] of computer scientists were women” and since then, the number has dropped to 22 percent.
This is an issue because under-representation leads to a narrow range of perspectives that diversity brings to research. In addition, the computer science field includes some of the fastest-growing and highest-paid jobs, therefore if this gap continues to widen, it could also impact and enlarge the wage gap.


\section{Technical Background}
I will start by creating a single web application game/code editor using JavaScript. Within my code editor, there will be a pre-coded game that’s mechanics resemble those of Flappy Bird. The user starts with a black ball as their avatar and black blocks instead of pipes. Every time the user loses the game, they’ll be asked the question: “What part of the game would you like to change?” They will be given the options of Avatar, Blocks/Obstacles, Background, or None to choose from, depending on the button they choose, the corresponding portion of the code will be highlighted (and along with it, more information on how to change it). In addition, I will also create a GitHub repository with images (.png files) that would aid users to customize their game to resemble Flappy Bird graphically. 


\section{Prior Work}
There’s a vast collection of computer science-related edutainment games that can be accessed through the internet. Most games help the user improve their skills or knowledge of a particular subject or language, whether it be the basics of programming (e.g. Move Code Lines or Comet 64) or something more fun and challenging (e.g. Screeps or Robocode). Some games even help people train their brains to practice logic-based thinking in order to properly communicate with computers. 
Through surfing the web, I’ve found a few games/projects that align with my goal of increasing the approachability of programming to a wider audience. An example of a coding edutainment game is Kodable. Kodable is geared towards elementary students, and it allows users to familiarize themselves with JavaScript by customizing their own “Fuzz” and taking them on coding adventures. Another similar project that I found was conducted by Girls Who Code in which the organization teamed up with rapper Doja Cat to make her music video for “Woman” user-interactive. They created an interface that lets viewers change images from the video using HTML, CSS, and JavaScript. 



\printbibliography 

\end{document}